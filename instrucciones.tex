\title{Installation Guide of PintOS}
\author{
		Prof: Jorge Gonzalez\\
			jgonzalez@utec.edu.pe\\
        TA: Martin Carrasco \\
         	martin.carrasco@utec.edu.pe
}
\date{\today}
\documentclass[12pt]{article}
\usepackage{hyperref}
\usepackage{listings}
\begin{document}

\maketitle

\section{Making a docker container for PintOS}

\begin{enumerate}
	\item Create the Dockerfile \textit{(Use the template Dockerfile included to guide you).}
	\begin{itemize}
		\item Pull the docker image from Docker Hub (e.g.: notice that in our template we pull the image of Ubuntu 14:04).
		\item Install dependencies (e.g.: qemu, build-essential, linux-headers-generic).
		\item Set enviromental variables (e.g: \texttt{export \$PINTOS\_HOME=\/pint-os\/}).
	\end{itemize}
	\item Create a volume to have data be persistant e.g. \textit{sudo docker volume create my\_volume PATH} (see additional info in \texttt{item 5}).
	
	\item Build the container e.g. \texttt{sudo docker build -t pintos .} ((dot) is part of the command).
	\item Run the container e.g. \texttt{sudo docker run -it --volume my\_volume:\/pint-os --name pint-sim pintos}. For more details about using volumes, please refer to \href{https://docs.docker.com/storage/volumes/}{Docker Documentation}.
	
\end{enumerate}

\section{Seting up and compiling PintOS}
\begin{enumerate}
	\item Make sure all the dependencies were installed correctly
	\item Compile the following submodules
	\begin{itemize}
		\item \texttt{userprog}
		\item \texttt{vm}
		\item \texttt{filesys}
	\end{itemize}
	\item Compile \textbf{/pint-os/utils} (\textbf{NOTE:} compile by using \texttt{make} on the folder).
	\item Only if there is an error related to \texttt{floor()} funcion, edit \textbf{/pint-os/utils/Makefile} to replace \texttt{CFLAGS= -lm} to \texttt{LDLIBS = -lm} then compile.
	\item Edit \textbf{/pint-os/thread/Make.vars} and change \texttt{SIMULATOR=} to \texttt{SIMULATOR= --qemu} 
	\item Compile in \textbf{/pint-os/threads}.
	\item Change \textbf{/pint-os/utils/pintos} \texttt{\$sim=bochos} to \texttt{\$sim=qemu} in \texttt{line 103}.
	\item Change \textbf{/pint-os/utils/pintos}, check \texttt{line 257}, \texttt{\$name = find\_file('kernel.bin')} to point to \textbf{/pint-os/threads/build/kernel.bin}
	%\item  Change \textbf{/pint-os/utils/pintos} \texttt{my (@cmd) = ('qemu')} to \texttt{my (@cmd) =qemu-system-x86\_64}
	\item Edit \texttt{line 362} of \textbf{/pint-os/utils/Pintos.pm}, \texttt{\$name = find\_file('loader.bin')} and point it to \textbf{/pint-os/threads/build/loader.bin}
	\item Go to folder \textbf{/pint-os/utils/} and test your program by executing: \texttt{./pintos run alarm-multiple}. The expected output is:

\begin{lstlisting}[basicstyle=\small, language=bash]
root@a9ecd025bd0e:/pint-os# pintos run alarm-multiple
qemu-system-i386 -device isa-debug-exit -hda /tmp/D86qNeoZCC.dsk -m 4 -net none -nographic -monitor null
PiLo hda1
Loading..........
Kernel command line: run alarm-multiple
Pintos booting with 4,088 kB RAM...
382 pages available in kernel pool.
382 pages available in user pool.
Calibrating timer...  258,457,600 loops/s.
Boot complete.
Executing 'alarm-multiple':
(alarm-multiple) begin
(alarm-multiple) Creating 5 threads to sleep 7 times each.
(alarm-multiple) Thread 0 sleeps 10 ticks each time,
(alarm-multiple) thread 1 sleeps 20 ticks each time, and so on.
(alarm-multiple) If successful, product of iteration count and
(alarm-multiple) sleep duration will appear in nondescending order.
(alarm-multiple) thread 0: duration=10, iteration=1, product=10
(alarm-multiple) thread 0: duration=10, iteration=2, product=20
(alarm-multiple) thread 1: duration=20, iteration=1, product=20
(alarm-multiple) thread 0: duration=10, iteration=3, product=30
(alarm-multiple) thread 2: duration=30, iteration=1, product=30
(alarm-multiple) thread 3: duration=40, iteration=1, product=40
(alarm-multiple) thread 0: duration=10, iteration=4, product=40
(alarm-multiple) thread 1: duration=20, iteration=2, product=40
\end{lstlisting}

	\item Optional: Add to the PATH the pintos binary folder using \texttt{export PATH=/pint-os/utils:\$PATH}. (already in Dockerfile)
	\item See \href{https://web.stanford.edu/class/cs140/projects/pintos/pintos_1.html}{PintOS Documentation} for more information. Notice that we are using Qemu.
\end{enumerate}
\end{document}



